%!TEX root = ./m_main.tex


\chapter{Заключение}

\section{Итоги работы}

В данной работе были получены следующие результаты:

\begin{itemize}
	\item Введена формальная обобщённая модель жизнедеятельности процессов в ОС для решения задачи восстановления
	\item Придуман и реализован алгоритм генерации промежуточного представления в виде графа действий, а так
	же генерации последовательности действий для восстановления дерева процессов
	\item На примере показана гибкость предложенного подхода, в сравнении с текущим подходом к восстановлению в системной утилите \texttt{criu}
	\item Написан набор программных утилит для визуализации промежуточных представлений, используемых генератором действий
\end{itemize}

Исходный код проекта можно найти по ссылке: \url{https://github.com/egorbunov/v2criu}.

\section{Дальнейшие планы}

Описанная в данной работе модель, как было показано в разделе~\ref{chap2:subsec:cycles}, обладает некоторыми недостатками: существует класс процессов, граф действий для которых не будет ациклическим. Как уже говорилось, одним из подходов к решению
этой проблемы является добавление вспомогательных временных процессов к дереву. Поэтому следующим шагом на пути к решению
задачи восстановления будет улучшение и усложенине приведённой модели.

Также, в силу того, что программу для восстановления в нашей модели мы представляем в виде графа, возможным развитием данной работы может стать параллельное исполнение такой программы из действий.
